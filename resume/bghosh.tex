\documentclass[a4paper,11pt,final]{article}
\usepackage[top=2cm,left=2cm,right=2cm,bottom=2cm]{geometry}% margins
\usepackage{graphicx}
\usepackage{helvet}
\renewcommand{\familydefault}{\sfdefault}
\pagestyle{empty}					% no pagenumbering
\setlength{\parindent}{0pt}			% no paragraph indentation
\usepackage{flowfram}										% column 										% figures
\usepackage{url}											% URLs
\usepackage[usenames,dvipsnames]{xcolor}					% color
\usepackage{multicol}										% columns 
\usepackage{tikz}
\setlength{\multicolsep}{0pt}
\usepackage{paralist}										% compact lists
\usepackage{tikz}
\setlength{\vcolumnsep}{\baselineskip}
\setlength{\columnsep}{\vcolumnsep}
\usepackage{enumerate}




\newcommand{\Sep}{\vspace{1.5em}}
\newcommand{\SmallSep}{\vspace{0.5em}}
\usepackage[hidelinks]{hyperref}

\begin{document}
	
	\begin{center}
		\Huge \textbf{Bishwamittra Ghosh}\\
		\vspace{0.2em}	
		\Large \centerline{ \textbf{Curriculum Vitae}}
	\end{center}
	
	\Sep
	\Large { \textbf{Personal Details}}\\
	\noindent\rule{\textwidth}{1pt}
	\normalsize 
	 \textbf{Contact Address:} Block 413, Sembawang Dr \#05-712, Singapore 750413\\
	\textbf{Email:} \url{bghosh@u.nus.edu}  \\
	\textbf{Phone:} +65 85990160\\
	\textbf{Date of Birth:}  28  November  1995 \\
	\textbf{Website: \url{https://bishwamittra.github.io/}} 
%	\textbf{Place of Birth:} Satkhira, Bangladesh\\
%	\textbf{Citizenship:} Bangladesh\\

	\Sep	
\Large { \textbf{Research Interest}}\\
\noindent\rule{\textwidth}{1pt}
\normalsize
My research interest is in explainable and fair machine learning using formal methods in Computer Science. 
	
	\Sep	
\Large { \textbf{Education}}\\
\noindent\rule{\textwidth}{1pt}
\normalsize
\textbf{National University of Singapore, Singapore}
\hspace*{\fill} 2018  - Present\\
Ph.D. candidate in  School of Computing\\

\SmallSep
\textbf{Bangladesh University of Engineering and Technology, Bangladesh}
\hspace*{\fill} 2013  -  2017\\
BSc. with Honors degree in Computer Science and Engineering\\


%\Sep
%\Large { \textbf{Research Experience}}\\
%\noindent\rule{\textwidth}{1pt}
%\normalsize
%\textbf{National University of Singapore, Singapore} \hspace*{\fill} 2018  - Present\\
%\begin{itemize}
%	\item Designed an incremental mini-batch training approach for learning interpretable classification rules in a MaxSAT-based framework
%	\item Designed an efficient combinatorial framework for learning a generalization of CNF/DNF classification rules
%	\item Formulated a MaxSAT-based framework for solving the combinatorial variant of group testing problem in collaboration with a Master's student
%	\item Presented published works in AIES-19, IJCAI-19, VLDB-19, and CP-19
%\end{itemize} 
%
%\SmallSep
%\textbf{Bangladesh University of Engineering and Technology, Bangladesh} \hspace*{\fill} 2016  - 2017\\
%\begin{itemize}
%	\item Designed a flexible socio-spatial group query to rank groups in the socio-spatial graph based on the social, spatial and group attributes
%\end{itemize} 


%\newpage
%\Sep
%\Large { \textbf{Awards}}\\
%\noindent\rule{\textwidth}{1pt}
%\normalsize
%\begin{itemize}
%	\item  \textbf{NUS Research Scholarship}\\
%	National University of Singapore
%	\item \textbf{Academic Merit Scholarship }
%	\\Dean's award,\\
%	Bangladesh University of Engineering and Technology
%	\item \textbf{Math Olympiad}\\
%	National and regional winner in higher secondary, secondary, and junior levels
%	\item \textbf{Scholarship}\\
%	Board scholarship in HSC, SSC, junior, and primary
%\end{itemize}


\Sep
\Large { \textbf{Selected Publications}}\\
\noindent\rule{\textwidth}{1pt}
\normalsize


	

	
	
\SmallSep

	\begin{enumerate}[{[}1{]}]
		\item {Justicia: A Stochastic SAT Approach to Formally Verify Fairness} \\
		\textbf{Bishwamittra Ghosh}, Debabrota Basu, Kuldeep S. Meel\\
		Proceedings of AAAI, 2021
		\item 	{A MaxSAT-based Framework for Group Testing} \\
		Lorenzo Ciampiconi, \textbf{Bishwamittra Ghosh}, Jonathan Scarlett, Kuldeep S. Meel\\
		Proceedings of AAAI, 2020
		\item {IMLI: An Incremental Framework for MaxSAT-Based Learning of Interpretable \\ Classification Rules}\\
		\textbf{Bishwamittra Ghosh}, Kuldeep S. Meel\\
		Proceedings of AIES, 2019
		\item 	{The Flexible Socio Spatial Group Queries}\\
		\textbf{Bishwamittra Ghosh}, Mohammed Eunus Ali, Farhana M. Choudhury,
		Sajid Hasan Apon, Timos Sellis, Jianxin Li\\
		Proceedings of the VLDB Endowment (PVLDB), 2019\\
		
	\end{enumerate}





%\SmallSep
%\textbf{Workshop papers}\\
%
%	\begin{enumerate}[{[W}1{]}]
%		\item 	{Interpretable Classification Rules in Relaxed Logical Form}\\
%		\textbf{Bishwamittra Ghosh}, Dmitry Malioutov, Kuldeep S. Meel\\
%		IJCAI workshop on XAI (Explainable Artificial Intelligence) and DSO (Data Science meets Optimization), 2019
%	\end{enumerate}

\Sep
\newpage
\Large { \textbf{Professional Experiences}}\\
\noindent\rule{\textwidth}{1pt}
\normalsize
\textbf{Conference Talks}\\
AAAI-2021, ECAI-2020, AAAI-2020, CP-2019, VLDB-2019, IJCAI-2019, AIES-2019

\SmallSep
\textbf{Research Fellowship}\\
 Max Planck Institute For Software Systems, Kaiserslautern, Germany \hspace*{\fill}2020

\SmallSep
\textbf{Graduate Assistant}\\
School of Computing, National University of Singapore \hspace*{\fill} 2018  - Present

\Sep
\Large{\textbf{Reference}}\\
\noindent\rule{\textwidth}{1pt}
\normalsize
\textbf{Kuldeep S. Meel}\\
Sung Kah Kay Assistant Professor\\
School of Computing\\
National University of Singapore\\

\textbf{Daniel Neider}\\
Research group leader\\
Max Planck Institute For Software Systems, Kaiserslautern, Germany
\end{document}
