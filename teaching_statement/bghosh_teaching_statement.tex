\documentclass[11pt]{article}

% Define margins
\setlength{\topmargin}{-1.0cm}
\setlength{\oddsidemargin}{0.1cm}
\setlength{\textwidth}{16.5cm}
\setlength{\textheight}{23.0cm}

% Packages
\usepackage{url}
\usepackage[hidelinks]{hyperref}
\usepackage{subfig}
\usepackage{graphicx}
\usepackage{comment}

% Color
\usepackage{xcolor}
\newcommand{\magenta}[1]{\textcolor{magenta}{#1}}
\newcommand{\red}[1]{\textcolor{red}{#1}}
\newcommand{\blue}[1]{\textcolor{blue}{#1}}
\newcommand{\green}[1]{\textcolor{green}{#1}}



\begin{document}
	\noindent\huge \textbf{Teaching Statement} \\
	\vspace{0.1em}\\
	\Large \textbf{Bishwamittra Ghosh}
		
	\normalsize
	\noindent Ph.D.\ Candidate\\
	School of Computing\\
	National University of Singapore (NUS)\\
	\blue{\url{https://bishwamittra.github.io}}



	\paragraph{}
		A key motivation for my pursuing an academic career is the enjoyment I find while teaching and mentoring students. My perspective of teaching is an alternate way to strengthen my understanding of concepts, propagate the joy of learning to young minds, and nurture brainstorming. I am fortunate to have experienced the teaching of great professors and lecturers during my undergraduate and graduate studies. I envision my academic career and teaching experience as a pathway to contribute to the community to whom I am indebted for my knowledge.
		
	
	\paragraph{Teaching Experience.}
		During my Ph.D., I have the opportunity to be a teaching assistant in two courses in NUS, Singapore. In CS2030: Programming Methodology II, I have developed scripts to grade students' Java codes. The focus was to automate the process of checking certain coding formats such as class dependencies and Java check styles. My generated report assists the course lecturer Henry Chia in the gradual improvement of students' coding skills. In another course, CS3243: Introduction to Artificial Intelligence, I together with fellow graduate students worked on designing course materials for Prof.\ Kuldeep S. Meel. I have learned the coordination of topics to be covered throughout a semester and how to divide them coherently so that students experience a gradual buildup of their understanding of artificial intelligence.
		
		
		Before joining the Ph.D., my lecturer position at UIU, Bangladesh has given me hands-on experience in teaching undergraduate students. I have lectured two courses: numerical methods and data communication. My duties include teaching a class of 30 students in each course, assigning semester projects, mentoring students, and grading exam papers.  I have enjoyed every aspect of teaching at UIU such as designing course materials in a comprehensible way for students of different maturity levels, solving tutorial questions with the active participation of students, designing mid-term and final-term questions, and grading them. Additionally, I have conducted two labs on artificial intelligence and microprocessor \& microcontrollers. In these labs, I have demonstrated software coding and interfacing microprocessors with hardware. Thus, my teaching experience has broadened my knowledge of CS theory and its applications and provided me with a great medium to interact with students. 
		
		
	\paragraph{Mentoring.}
	
		During my course of Ph.D., I am fortunate to have worked with fellow graduate and master's students on several research problems. Lorenzo, an exchange master's student in NUS, and I have worked on proposing a MaxSAT-based formulation for solving the classical group testing problem. I have been involved with Lorenzo in designing experiments, writing the paper while mentoring Lorenzo different approaches for presentable paper writing, and communicating among co-authors for submitting to AAAI, a premier conference in artificial intelligence. Our paper got accepted in AAAI-2020, and it constituted the core contribution of Lorenzo's master's thesis.
		
		In another research study on MaxSAT problems within a streaming model, I have mentored Mohimenul, a Ph.D. student at NUS. We have discussed different aspects of formulating the problem, working on the solution, and designing experiments. In addition, I have provided feedback on Mohimenul's writings so that Mohimenul can become an independent researcher. Finally, we have submitted the paper in AAAI-2022. Upon publication, this paper will contribute to Mohimenul's Ph.D. thesis.

	
	\paragraph{Future Courses.}
	
		I am qualified in teaching machine learning and artificial intelligence, probability theory and statistics, and introduction to logic and formal methods to undergraduate and master's students. Machine learning has witnessed a significant success in the last decades through applications in different safety-critical domains. To this end, I aim to teach trustworthy machine learning techniques, particularly the fairness and explainability aspects of machine learning based on automated reasoning, formal methods, and statistics. To conclude, my teaching philosophy lies in introducing students to recent research findings and connecting them to the fundamental concepts in computer science.
		
		
\end{document}